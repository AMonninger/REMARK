\documentclass{beamer}
\usetheme{default}
\providecommand{\EqDir}{../Equations}

\title{Replication of Aiyagari(1994)}
\author{Zixuan Huang and Mingzuo Sun}
\begin{document}
\begin{frame}[plain]
    \maketitle
    
    
\end{frame}

\begin{frame}{Individual's Problem}
	
	\begin{align}
	\max E_0\left(\sum_{t=0}^\infty \beta^t U(c_t)\right)\\
	\text{s.t.}\\
	c_t+a_{t+1}=wl_{t}+(1+r)a_t \\
	c_t\geq0\\
	a_t\geq-\phi
\end{align} \\
	where $\phi$ (if positive) is the limit on borrowing; $l_t$ is assumed to be i.i.d with bounded support given by $[l_{min},l_{max}]$, with $l_{min}>0$; $w$ and $r$ represent wage and interest rate respectively.
	
\end{frame}

\begin{frame}{Bellman Equation and Euler Equation}
	The Bellman equation is as follows:
	\begin{align}
$k_{jt+1}=(1-\delta)k_{jt}+i_{jt}$
\end{align}
	
	Consequently, Euler equation is: 
	\begin{align}
	1= R\beta E_{t-1}[\{c_t[R[m_{t-1}-c_{t-1}]/Gn_t + v_t]Gn_t/c_{t-1}\}^{-\rho}]
\end{align}
\end{frame}

\begin{frame}{Solve the Model}
	The decision rule can be written as: 
	\begin{align}
	\hat{a}_{t+1}=A(z_t,\phi,w,r)
	\end{align}
	
	And the law of transition would be:
	\begin{align}
	z_{t+1}=wl_{t+1}+(1+r)A(z_t,\phi,w,r)-r\phi
	\end{align}
	
\end{frame}

\begin{frame}{Firm's Problem}
	\begin{align}
	\max F(K,L)-wL-rK
\end{align}
	where $K$ is the aggregate capital, $L$ is the aggregate labor, $F(K,L)$ is the production function.
\end{frame}

\begin{frame}{Computation and Tools}
	\begin{itemize}
		\item We used $\texttt{Dolo}$ and $\texttt{Dolark}$ to solve the model.
		\item Codes are written in Python. 
		\item You can easily find details of our replication on $\hyperlink{www.github.com/econ-ark/REMARK}{EconArk\ GitHub \ page}$.
	\end{itemize}
	
	
\end{frame}

\end{document}
